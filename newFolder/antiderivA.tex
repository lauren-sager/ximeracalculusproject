\documentclass{ximera}

\title{Antiderivative Activity - Acceleration Variant}  %\xmalt{} for alt text
\author{MATH 425: Calculus I}

\begin{document}
\begin{abstract}
  Working with the peers in your group, solve the following problems. Make sure to show and justify all your work. Make sure everyone in the group understands the solution and participates. Be prepared to report your answers to the whole class. 
\end{abstract}
\maketitle



\begin{exercise}
    A car travels at $51.333 \frac{ft}{s}$ when the gas pedal is applied, causing a constant acceleration of $17\frac{ft}{s^2}$.  Assume that the moment the gas pedal is applied is time $0$.  
    \begin{enumerate}
        \item[A.] On the graphs below, sketch a graph that show the acceleration of the car with respect to time. (Hint: The $x$-axis on the graphs should be time, which we will call $t$ instead of $x$.) 
        \item[B.] On the graphs below, sketch a graph the shows the velocity of the car with respect to time. 
        \item[C.] Sketch the \textit{shape} of the graph that shows the position (or location, not to be confused with distance traveled) of the car with respect to time. (To simplify your graph, assume that at the time the gas pedal is applied, the position of the car is 0.) \\
        \begin{image}
            \includegraphics{AntiderivSMMPtask1.png}
           
        \end{image}
        A. Acceleration B. Velocity C. Position
        \item[D.] Find an equation for the velocity $v(t)$ of the car at time $t$.  Does your sketch in part B match your equation?  (Check: What are the units of the result?) 
        \item[E.] Find an equation for the position $s(t)$ of the car at time $t$, assuming that the car starts at position 0 at time $t=0$. Does your sketch in part C (approximately) match your equation?  (Check: What are the units of the result?)
        \item[F.] Using your equations from part D and E, determine how far the car travels by the time when the velocity has reached $100\frac{ft}{s}$.
    \end{enumerate}
\end{exercise}
\begin{exercise}
    A ball is tossed vertically in such a way that its velocity function is given by $v(t)=32-32t$, where $t$ is measured in seconds and $v$ in feet per second. Assume that this function is valid for $0\leq t\leq2$.
    \begin{enumerate}
        \item[A.] For what values of $t$ is the velocity of the ball positive?  What does this tell you about the motion of the ball on this interval of time values?
        \item[B.] Find an antiderivative (indefinite integral), $s$, of $v$ that satisfies $s(0)=0$.
        \item[C.] Compute the value of $s(1)-s(\frac12)$.  What is the meaning of the value you find?
        \item[D.] Using the graph of $v(t)$ provided below, find the exact area under the velocity curve between $t=\frac12$ and $t=1$.  What is the meaning of the value you find? 
        \begin{image}
            \includegraphics{AntiderivSMMPtask2.png}
        \end{image}
        \item[E.] Answer the same questions as in C and D, but instead using the interval $[0,1]$.
        \item[F.] What is the value of $s(2)-s(0)$?  What does this result tell you about the flight of the ball?  How is this value connected to the provided graph of $v(t)$? \\
        (Problem adapted from \textit{Active Calculus} 1st ed. by Boelkins, Austin and Schlicker.)
    \end{enumerate}
\end{exercise}
    




\end{document}
