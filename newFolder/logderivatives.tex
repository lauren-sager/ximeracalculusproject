\documentclass{ximera}

\title{Logarithmic Derivatives Activity}
\author{MATH 425: Calculus I}

\begin{document}
\begin{abstract}
    Working with the peers in your group, solve the following problems. Make sure to show and justify all your work. Make sure everyone in the group understands the solution and participates. Be prepared to report your answers to the whole class. 
\end{abstract}
\maketitle


\begin{exercise}
    The mass of radioactive elements decaying in a sample can be modeled by 
      $$m(t)=m_0e^{kt},$$
    where $k$ is a negative constant and $m_0$ signifies the initial mass of the sample.  Strontium-90 has a half life of 28 days.  A sample of Strontium-90 has and intial mass of 50 mg.
    \begin{enumerate}
        \item Explain the meaning of ``a half-life of 28 days'' in this context.
        \item Determine the constants $m_0$ and $k$ for this Strontium-90 sample.  Then find an equation for the mass remaining after $t$ days.  (This equation is a decay function.)
        \item How long does it take the sample to decay to a mass of 2mg?
        \item Take the derivative of your decay function from part b.  What does this derivative represent?  Explain.
        \item Evaluate the derivative of the decay function at the time from part c.  What does the result represent?
    \end{enumerate}
\end{exercise}

\begin{exercise}
    A logistic growth model of population growth reflects the fact that some populations cannot grow without bound, such as populations with limited resources like space and food.\\
    This equation give an example of a simple logistic growth model:
      $$P(t)=\frac{MP_0}{P_0+(M-P_0)e^{-Mt}}, \,\,\,\,\,\,\,\, M,P_0 \text{ with } M>P_0,$$
    where $t$ is measured in years and has domain $[0,\infty)$.
    \begin{enumerate}
        \item To understand what the constant $P_0$ represents, evaluate $P(t)$ at $t=0$.  In other words, find $P(0)$.
        \item To understand what the constant $M$ represents, find $\lim_{t\rightarrow\infty}P(t)$.  Why do you think we often refer to the constant $M$ as the ``carrying capacity'' of the model?
        \item Compute $P'(t)$.
        \item What would $P'(t)$ equal if $P(0)=M$? Does this make sense given what the model represents?
        \item Write a story about a population that might be described by this model.
        \item What would happen if $P_o\geq M$?
    \end{enumerate}
\end{exercise}



\end{document}
