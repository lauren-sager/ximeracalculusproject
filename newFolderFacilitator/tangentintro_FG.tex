\documentclass{ximera}

\title{Intro to Tangent Lines Activity Facilitator Guide}
\author{MATH 425: Calculus I}

\begin{document}
\begin{abstract}
    Working with the peers in your group, solve the following problems. Make sure to show and justify all your work. Make sure everyone in the group understands the solution and participates. Be prepared to report your answers to the whole class. 
\end{abstract}
\maketitle

\textbf{Student Learning Objectives}
Students will be able to:
\begin{itemize}
  \item Calculate average rate of change given a function or points.
  \item Interpret average rate of change given a context.
  \item Find the equation of a secant line to a function.
\end{itemize}
\textbf{Prerequise Knowledge:} This activity assumes that students have been introduced to the average rate of change equation $\frac{f(b)-f(a)}{b-a}$ for $f$ from $a$ to $b$.  One question asks students to find average rate of change over successively smaller intervals and use this to extrapolate an instantaneous rate of change.  We try to avoid many of these calculations, in order to prevent students as seeing this as an alternative to using the derivative definition. Exercise 4 tries to anticipate the definition of the derivative. \\
\\
This is an activity where it's very hard to judge how quickly students will move through it.  The final exercise is more challenging, so it is meant mostly as a final exercise to provide challenge to those groups who fly through the rest of the activity.  Not every group need get through the final problem.  The part (f) is particularly challenging and can have different paths to a viable answer.  Part (e) may benefit from divide and conquer tactics, as it is a lot of straight calculation.

\begin{exercise}
If a ball is thrown upwards into the air with a velocity of 40 ft/s, its height in feet $t$ seconds later is given by $y=40t-16t^2$.  
 Find the average velocity for the time period beginning at 2 seconds and lasting 
    \begin{enumerate}
        \item 0.5 seconds\\
        \textcolor{blue}{ $\frac{f(2.5)-f(2)}{2.5-2}=\frac{0-16}{.5}=-32$}
        \item 0.1 seconds\\
        \textcolor{blue}{ $\frac{f(2.1)-f(2)}{2.1-2}=\frac{13.44-16}{.1}=-25.6$} 
        \item 0.05 seconds\\
        \textcolor{blue}{ $\frac{f(2.05)-f(2)}{2.05-2}=\frac{14.76-16}{.05}=-24.8$}
        \item 0.01 seconds\\
        \textcolor{blue}{ $\frac{f(2.01)-f(2)}{2.01-2}\approx \frac{15.76-16}{.01}=-24$} 
    \end{enumerate}
Estimate the instantaneous velocity of the same ball when $t=2$ using your work from question 1.  Explain what your answer says about what is happening to the ball at $t=2$. \\
\textcolor{blue}{ The average velocity at $t=2$ is around -24 ft/s, meaning that the ball is coming back downwards at a rate of 24 feet per second. }
\textcolor{orange}{Try to avoid reinforcing students' idea that this is a viable way to find the exact instantaneous rate of change. Point out the \textit{estimate} part of the question if students seem to be relying on this as an exact answer.  Some students will also know these formulas (and potentially even the instantaneous rate of change formula) from physics.  Encourage them to use the mathematical means they've learned rather than just pulling out the physics formulas. \textit{Checking} their work with these formulas is encouraged!}

\end{exercise}

\begin{exercise} 
A secant line to the function $f$ is a line through the points $(x_1, f(x_1))$ and $(x_2, f(x_2))$ on the graph of $f$.  It must intersect with the graph at least twice at distinct points.  Find the slope of the secant line to $f(x)=40x-16x^2$ using the points $x=2$ and $x=2.5$. (Hint: look at your calculations in exercise 1).  Does this seem familiar?) \\
\textcolor{blue}{ $\frac{f(2.5)-f(2)}{2.5-2}=\frac{0-16}{.5}=-32$} 
\end{exercise}

\begin{exercise}
  Find the equation of the secant line to the function $f(x)=\frac{x+4}{x-2}$ through the points at $x=-2$ and $x=4$.\\
  \textcolor{blue}{ $f(-2)=\frac{-2+4}{-2-2}=\frac{2}{-4}=-\frac12$\\
    $f(4)=\frac{4+4}{4-2}=\frac{8}{2}=4$\\
    $m=\frac{4+\frac12}{4+2}=(\frac{9}{2})(\frac16)=\frac34$\\
    Line equation: $y-4=\frac34(x-4)$\\
    or $y=\frac34x+1$\\
    or $y+\frac12=\frac34(x+2)$}
\end{exercise}

\begin{exercise}
  The position function for a falling ball is given by $s(t)=16-16t^2$, where $s$ is measured in feet and $t$ in seconds.
  \begin{enumerate}
    \item Find an expression for the average velocity of the ball on a time interval of the form $[0.5, 0.5+h]$ where $0<h<0.5$, or $[0.5+h,0.5]$ when $-0.5<h<0$ and $h\neq 0$.\\
    \textcolor{blue}{Average velocity=$\frac{f(0.5+h)-f(0.5)}{0.5+h-(0.5)}=\frac{16-16(0.5+h)^2-16+16(0.5)^2}{h}$
    $=\frac{16-4-16h-16h^2-16+4}{h}=\frac{-16h-16h^2}{h}=\frac{h(-16-16h)}{h}$\\
    $=-16-16h$}\\
    \textcolor{orange}{Many students inevitably find trying to incorporate variables into these types of calculations challenging.  This looks ahead to the definition of the derivative, so encourage students to work slowly and carefully through the calculation.  They have all the tools!}
    \item Use this expression to compute the average velocity on $[0.5, 0.75]$ and $[0.4, 0.5]$.\\
    \textcolor{blue}{On $[0.5,0.75]=[0.5,0.5+0.25]$, $h=0.25$\\
    Average velocity$=-16-16(0.25)=-16-4=-20$ ft per sec\\
    On $[0.4,0.5]=[0.5-0.1,0.5]$, $h=-0.1$\\
    Average velocity$=-16-16(-0.1)=-16+1.6=-14.4$ ft per sec}
    \item Make a conjecture about the instantaneous velocity at $t=0.5$. \\
    \textcolor{blue}{We can't determine the instantaneous velocity at $t=0.5$ exactly, but we can conjecture that it is between -20 ft/sec and -14.4 ft/sec.}
  \end{enumerate}
  Problem from Boelkins, Austin \& Schlicker (2018) \textit{Active Calculus 2.1}.
\end{exercise}

\begin{exercise}
    According to the U.S. census, the population of the city of Grand Rapids, MI, was 181,843 in 1980; 189,126 in 1990; and 197,800 in 2000.
    \begin{enumerate}
        \item Between 1980 and 2000, by how many people did the population of Grand Rapids grow?\\
        \textcolor{blue}{$197800-181843=15957$ people}
        \item In an average year between 1980 and 2000, by how many people did the population of Grand Rapids grow?\\
        \textcolor{blue}{Average change per year: $\frac{15957}{20}=797.85$ people per year}
        \item Just like we can find the average velocity of a moving body by computing change in position over change in time, we can compute the average rate of change of any function $f$.In particular, the average rate of change of a function $f$ over an interval $[a,b]$ is the quotient
          \[\frac{f(b)-f(a)}{b-a}.\]
        What does the quantity  $\frac{f(b)-f(a)}{b-a}$ measure on the graph of $y=f(x)$ over the interval ?\\
        \textcolor{blue}{This measures the slope of the secant line on the graph through the points $(a,f(a))$ and $(b,f(b))$.}
        \item Let $P(t)$ represent the population of Grand Rapids at time $t$, where $t$ is measured in years from January 1, 1980. What is the average rate of change of $P$ on the interval from $t=0$ to $t=20$? What are the units on this quantity?\\
        \textcolor{blue}{This is the average rate of change we already calculated: $\frac{P(20)-P(0)}{20-0}=797.85$ people per year.}
        \item If we assume the population of Grand Rapids is growing at a rate of approximately 4\% per decade, we can model the population function with the formula
        \[P(t)=181843(1.04^{\frac{t}{10}}).\]
        Use this formula to compute the average rate of change of the population on the intervals $[5,10]$, $[5,9]$, $[5,8]$, $[5,7]$ and $[5,6]$.\\
        \textcolor{blue}{For $[5,10]$:\\
        $\frac{P(10)-P(5)}{10-5}=\frac{189116.72-185444.20}{5}=734.50$ people per year\\
        For $[5,9]$:
        $\frac{P(9)-P(5)}{9-5}=\frac{188376.44-185444.20}{4}=733.06$ people per year\\
        For $[5,8]$:\\
        $\frac{P(8)-P(5)}{8-5}=\frac{187639.06-185444.20}{3}=731.62$ people per year\\
        For $[5,7]$:\\
        $\frac{P(7)-P(5)}{7-5}=\frac{186904.57-185444.20}{2}=730.18$ people per year\\
        For $[5,6]$:\\
        $\frac{P(6)-P(5)}{6-5}=\frac{186172.94-185444.20}{1}=728.74$ people per year\\}
        \textcolor{orange}{This is a prime example of a good divide and conquer group problem.  Each student can take one interval and do the calcuation.}
        \item How fast do you think the population of Grand Rapids was changing on January 1, 1985? Said differently, at what rate do you think people were being added to the population of Grand Rapids as of January 1, 1985? How many additional people should the city have expected in the following year? Why?\\
        \textcolor{blue}{It seems like these values are changing by (very approximately) 1.5 people per year.  This is an exponential function, so it does not grow at a constant rate, but we can use this large grained approach to make a broad estimation.  If our assumption holds, it seems as though Grand Rapids should expect to be growing at a rate of about 727.25 people per year, or expect their population to grow about this many people in the following year.}
        \textcolor{orange}{This is a somewhat vague question. There are many viable ways to reason about what the rate might look like, and the point is that students should be able to defend their reasoning and/or assumptions.  Using a derivative (not expected from students at this point), the actual rate is about 727.33 people per year.}
  \end{enumerate}
  Problem from Wakefield, et al., \textit {Coordinated Calculus}.
\end{exercise}



\end{document}