\documentclass{ximera}

\title{First Day Activity}
\author{MATH 425: Calculus I}

\begin{document}
\begin{abstract}
    Working with the peers in your group, solve the following problems. Make sure to show and justify all your work. Make sure everyone in the group understands the solution and participates. Be prepared to report your answers to the whole class. 
\end{abstract}
\maketitle

\textbf{Student Learning Outcomes}
Students will be able to:
\begin{itemize}
    \item Reflect on the processes that group work requires.
\end{itemize}

\textbf{Prerequisite Knowledge:} This activity purposely does not have any mathematical prior knowledge required.  Students may interact with the Fermi problems at many different levels, based on their comfort level, prior knowledge and math background.  The important part here is to get the groups working together on something that will go much better if multiple students work together than if they all work alone.  \\

I tried to provide some research-backed outcomes improved by group/cooperative learning at the end of the activity.  The article they come from is helpful, if you are interested in reading (or skimming) it.

\begin{enumerate}
    \item What is the name you would like to be called?  You can also include your pronouns.
    
    \item Introduce yourself to your group.  As a group, find three ``non-trivial'' things (not things like the fact you are in this class, that you go to UNH, that you are a student in CEPS, etc.) that you have in common.  Once your group has your three things, call over your LA or TA to see if they share any of your common things. \\
    \vspace{.5in}
    \item In this activity you will investigate a ``Fermi Problem.''  Enrico Fermi was a Italian physicist who was famous for work on atomic energy and the Manhattan project, along with a knack for closely estimating mathematical answers to problems with little to no data.\\
    Fermi problems are a family of problems which are extremely difficult, inefficient or impossible to solve exactly.  They generally will require some assumptions, but based on those assumptions, a reasonable answer can be arrived at. This answer can be used to check a time- and effort-intensive exact solution, or to indicate assumptions which may be made about the exact answer.  

    Along these lines, your final answer matters less here than how you arrived at it, and what assumptions you made along the way.  Discuss with your group, and come up with an estimation you'd be willing to share with the class.  Be prepared to share your process.  

    Additionally, think about the process of working with a group.  Part of the discussion to follow this activity will involve thinking about norms for group work.  
\end{enumerate}
\begin{exercise}
\textbf{Problem: How long does it take to count to a million?}
\end{exercise}

\begin{exercise}
    \textbf{Bonus Problem: How many paperclips would fit in this room?}
\end{exercise}

Group work and cooperative learning will be an important part of recitation in this course.  Why group work? \\
Research shows that group work leads to:
\begin{itemize}
    \item higher individual achievement for students working in groups
    \item increased motivation, creativity in problem solving and persistance in students who work in groups
    \item improved interpersonal relationships in the classroom (... remember you'll be working in groups in your career, as well)
    \item improved self-esteem and social skills
\end{itemize}
(Johnson, et al., 1998).\\

While not as difficult to solve as Fermi problems, many of the problems you will work on in recitation will be built with the understanding that you will be working on groups.  This means they may be more challenging than problems we ask you to solve on your own - after all, you have multiple heads working on the problem instead of just one!    \\

Not every group work experience will be positive, unfortunately.  You may even have prior negative associations with group work.  We ask that you give it a try.  We are not arbitrarily asking you to work together; the research really does say it improves outcomes!  \\
\\
References:\\
Johnson, D. W., Johnson, R. T., Smith, K. A., Lombardi, V., Coach, F., \& To, C.
L. R. (1998). Cooperative learning returns to college: What evidence is there
that it works? \textit{Change: The Magazine of Higher Learning}, 30 (4), 26–35.



\end{document}