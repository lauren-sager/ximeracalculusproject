\documentclass{ximera}

\title{Optimization Activity}
\author{MATH 425: Calculus I}

\begin{document}
\begin{abstract}
    Working with the peers in your group, solve the following problems. Make sure to show and justify all your work. Make sure everyone in the group understands the solution and participates. Be prepared to report your answers to the whole class. 
\end{abstract}
\maketitle


\begin{exercise}
    A piece of 11"x13" paper is being made into a box without a top.  The box is formed by cutting equal-sized squares (the corner with side length $x$) from each paper corner and folding the sides up.  (see figure 1).  
    \begin{image}
    \includegraphics{optimizationEx1.png}
    \end{image}
    \begin{center}
        Figure 1.\\
    \end{center}
    \begin{enumerate}
        \item Describe the relationships between the corner $x$ and the box's volume in words.  For example, how do you expect the volume to change based on the size of $x$?\\
        \textcolor{blue}{This is a more complicated relationship than it may seem.  As the corner gets bigger, the box gets deeper.  However, the bottom of the box then gets smaller.  For small $x$, the volume will increase as $x$ grows, but this is not consistent across all values of $x$, as can be seen in the applet.}
        \item Explore this Geogebra applet: (\href{https://www.geogebra.org/classic/nypGxGTg}{https://www.geogebra.org/classic/nypGxGTg})
        \begin{center}
          \geogebra{nypGxGTg}{1200}{600}
        \end{center}
        Keep track of three things you wondered and three things you noticed, and write them below.
        \begin{tabular}{l|l}
            We wondered ... & We noticed ...\\ \hline
            (1) & (1)\\
            (2) & (2)\\
            (3) & (3)\\
        \end{tabular}
        \item Explore the Geogebra applet again with paper dimensions 11"x13".  What does the graph reprsent?\\
        \textcolor{blue}{The graph represents the volume of the box, with respect to the corner size ($x$).}
        \item What do the $x$-axis and $y$-axis represent? \\
        \textcolor{blue}{The $x$-axis represents the length of the corner sides, while the $y$-axis respresents the volume of the box.}
        \item How is the volume formula constructed?  What is the domain?\\
        \textcolor{blue}{\[V(x)=x(13-2x(11-2x))\]
        The volume is the length after cutting out the corner $(13-2x)$ times the width after cutting out the corner $(11-2x)$ time the height $x$ (comes from the amount folded up).\\
        The length of $x$ cannot be longer than half the shortest side: 5.5 inches, and cannot be less than 0.}
        \item What are the critical points on the graph?\\
        \textcolor{blue}{
            \[V(x)=x(11-2x)(13-2x)=(11x-2x^2)(13-2x)=4x^3-48x^2+143x\]
            \[V'(x)=12x^2-96x+143\]
            \[12x^2-96x+143=0\]
            \[x=\frac{96\pm\sqrt{(-96)^2-4(12)(143)}}{2(12)}\approx 1.98,6.02\]
            $6.02$ is not in the domain of the function (this would create a corner longer than half the 11'' side). \\
            The endpoints $x=0,5$ are also critical points.\\
            So all critical points are $x=0,1.98,5.5$.
        }
        \item Classify the critical points as local maximum, minimum or neither.\\
        \textcolor{\[V(0)=0\]
        \[V(1.98)=126.01\]
        \[V(5.5)=0\]}
        \item What does the maximum volume mean in terms of context?\\
        \textcolor{This is the largest box that will hold the most volume ($126.01in^2$) created from the given paper.}
    \end{enumerate}
    Adapted from Moore \& Carlson (2012) \textit{Students' Images of Problem Contexts when Solving Applied Problems}.
\end{exercise}

\begin{exercise}
    A farmer has 120 m of fencing to enclose a rectangular pumpkin patch along an irrigation canal, where no fence will be needed.
    \begin{enumerate}
        \item Sketch the problem by labeling its dimensions with appropriate variables.
        \item Use your variable to determine an expression for the fenced area.
        \item Sketch the graph of the function representing the area.
        \item What are the dimensions for the patch with the largest area, and what is this area?
    \end{enumerate}
\end{exercise}

\begin{exercise}
    A soup can in the shape of a right circular cylinear is made from two materials.  The material for the side of the can costs \$0.015 per square inch, and the material for the ends (lids) costs \$0.027 per square inch.  Suppose that we want to construct a can with a volume of 16 cubic inches.  What dimensions minimize the cost of the can? 
    \begin{enumerate}
        \item Draw a picture of the can and label its dimensions with appropriate variables.
        \item Use your variables to determine an expression for the volume, surface area, and cost of the can.
        \item Determine the total cost function as a function of a single variable.
        \begin{enumerate}
            \item What is the domain on which you should consider this function?
            \item Sketch the graph of the total cost function.  You may use a graphing calculuator or Desmos. 
        \end{enumerate}
        \item Find the absolute minimum cost and the dimensions that produce this value.
    \end{enumerate}
    Adapted from Boelkins, Austin \& Schlicker (2018) \textit{Active Calculus 2.1}.
\end{exercise}

\begin{exercise}
    Reflect on these problems.  What do they have in common and what is different?  What would you say to help someone who is going to solve these problems? \\
    This type of reflection can help you make connections and feel more confident when problems that don't look exactly the same as the problems you have done show up in other contexts.
\end{exercise}




\end{document}
